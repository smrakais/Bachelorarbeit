\chapter{Zusammenfassung und Ausblick}
Die Photolumineszenz der verwendeten Halbleiterprobe, kann mit Hilfe 
eines äußeren angelegten Magnetfelds gerichtet werden.
Dieser Effekt wird als transversale magnetische Führung (TMRLE)
bezeichnet.
Diese Steuerung des Lichts wird im Experiment durch die temperaturabhängige Größe der Direktionalität $C$
quantifiziert. 

Die Direktionalität wurde über einen Temperaturbereich von \SI{41}{\kelvin}
untersucht.
Der gemessene Maximalwert der Direktionalität, bei einer eingestellten Temperatur
von \SI{4}{\kelvin} beträgt \SI{13}{\percent} (vgl. Abbildung~\ref{fig:temp_all_nach}). 
Bei einer Erhöhung der Temperatur konnte eine Verringerung der Direktionalität $C(T)$ festgestellt werden.
Dieser Zusammenhang bestätigt die zugrunde liegende Theorie.
Die größte Abschwächung hat die Direktionalität bei einer Temperatur von \SI{45}{\kelvin}
erfahren.
Hierbei sank die Direktionalität bei einem Anfangswert von $C=\SI{13}{\percent}$ bei \SI{4}{\kelvin} 
auf einen Wert von $C=\SI{4,8}{\percent}$ bei \SI{45}{\kelvin} ab.

Die Funktion~\ref{eq:C} lieferte eine gute Approximation des Verlaufs der Direktionalität $C(T)$
in Abhängigkeit der Temperatur (vgl. Abbildung~\ref{fig:temp_all_nach}).
Um die Genauigkeit des Fits zu steigern, könnte in weiterführenden Experimenten 
die Temperaturabhängigkeit über einen größeren Bereich und mit mehreren Messwerten untersucht werden.
Ein weiterer Punkt ist, dass durch den Fit ein relativ hohes $T_\text{off} =  \SI{20\pm 2}{\kelvin}$
entstanden ist, was unter anderem daran liegt, dass der Temperatursensor nicht direkt auf der Probe platziert ist.
Die Platzierung des Temperatursensors könnte in einem neuen Experiment optimiert werden.

Der verwendete Laser hatte im Experiment eine Leistungsdichte von \SI{400}{\watt\centi\meter^{-2}}.
Interessant zu wissen wäre, ob oder welchen Einfluss eine abweichende Leistungsdichte 
auf die Direktionalität hat, denn der Laser sorgt ebenfalls für eine Erwärmung der Probe.

Ein weiterer Aspekt, der genauer untersucht werden könnte, ist der Aspekt der Gitterperiode.
Im Experiment lag diese bei einem Wert von \SI{250}{\nano\meter}.
Durch eine Veränderung der Gitterperiode lässt sich die Plasmonresonanz durch das Spektrum bewegen, 
wodurch sich zum Beispiel bei gleichbleibender Lichtwellenlänge die Winkelabhängigkeit der Direktionalität verändert. 
Die Untersuchung abweichender Perioden könnte zum Beispiel für ein genaueres Verständnis über das 
Zusammenspiel der verschiedenen physikalischen Vorgänge sorgen.
