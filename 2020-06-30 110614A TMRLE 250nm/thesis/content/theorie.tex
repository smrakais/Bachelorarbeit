\chapter{Theoretische Einführungen}

\section{Oberflächenplasmonenpolaritonen}~\label{sec:spps}
Oberflächenplasmonenpolaritonen (engl. surface plasmon polaritons, SPPs)
sind kollektive Elektronen Oszillationen welche sich entlang der Grenzfläche
zwischen Leiter und Dielektrikum ausbreiten.\cite{plasmonics}
Sie entstehen durch Anregungen des Elektronengases in metallischen Materialien.
Eine Anregung kann z.B. durch Laserlicht geschehen.

\begin{wrapfigure}{l}{5cm}
    \includegraphics[width=4.5cm]{./Plots/leiter_und_nichtleiter.pdf}
    \caption{Darstellung einer Grenzfläche zwischen Leiter und Dielektrikum an welcher 
    SPPs entstehen können. Zur Vereinfachung hier eine Ausbreitung in x-Richtung angenommen.\\}
    \label{fig:kasten}
\end{wrapfigure}
\FloatBarrier

Die kollektiven Oszillationen können als Quasiteilchen behandelt werden. 
Also als ein Phänomen was ein Teilchencharakter besitzt bzw. einem ein solcher zugeordnet werden kann. 
Eine Eigenschaft der SPPs ist ihr evaneszentes Verhalten bezogen auf das
durchdringen der Grenzfläche.
Der theoretische Zugang zu SPPs gelingt über die Verwendung der Maxwellgleichungen, welche dazu
genutzt werden um die Helmholzgleichung 
\begin{equation}
    \nabla^2 \vec{E} + k^2_0 \, \epsilon \, \vec{E} = 0
    \label{eq:helm}
\end{equation}
zu formulieren. 
Dabei steht $\vec{E}$ für den elektrischen Feldvektor, $k_0 = \sfrac{\omega}{c}$ 
für der Wellenvektor und $\epsilon$ für die Permittivität.
Aus Gleichung~\ref{eq:helm} lassen sich im folgenden mehrere gekoppelte Differentialgleichungen
herleiten, welche wiederrum in 2 Gleichungen überführt werden, die Transversalelektrischewellen (TE) und 
Transversalmagnetischewellen (TM) beschreiben.
Die Besonderheit bei TM-Wellen und TE-Wellen ist das 
die Komponente der Amplitude in Ausbreitungsrichtung verschwindet.

Werden TE-Wellen betrachtet sorgen die Randbedingungen an der Grenzfläche
dafür das die Amplitude insgesamt verschwindet.\cite{plasmonics} 
Aus diesem Grund gibt es nur SPPs, welche durch TM-Moden beschrieben werden. 
Die Dispersionsrelation für eine propagierende Welle von SPPs lautet
\footnote{Hier wird der Fall einer sich in x-Richtung Ausbreitenden Welle betrachtet.}
\begin{equation}
    k_\text{spp} = k_0 \sqrt{\frac{\epsilon_1 \,\epsilon_2}{\epsilon_1 + \epsilon_2}}.
    \label{eq:disp}
\end{equation}
Hierbei sind $\epsilon_1$ und $\epsilon_2$ die unterschiedlichen Permittivitäten des Leiters und
des Dielektrikums.
Wie bereits genannt lassen sich SPPs durch Laserslicht anregen. 
Um dies zu realisieren muss sich ein metallisches Gitter (z.B. aus Gold) auf der Oberfläche befinden.

\begin{figure}
    \centering
    \begin{subfigure}{0.5\textwidth}
        \centering
        \includegraphics[width=5.5cm]{./Plots/disp.png}
        \caption{Darstellung der Dispersionsrelation von einer SPP-Welle und der eines freien Photons.
        Es existiert für keinen Wert von $k_x$ ein Schnittpunkt der Dispersionen. \cite{disp}}
        \label{fig:disp}
    \end{subfigure}
    \begin{subfigure}{0.5\textwidth}
        \centering
        \includegraphics[width=5.5cm]{./Plots/gitter.png}
        \caption{Periodische Gitterstruktur welche sich auf einer Probenoberfläche befindet. 
        Diese ist notwendig damit die Plasmonen in der Lage sind an Photonen zu koppeln.\cite{disp} }
        \label{fig:gitter}
    \end{subfigure}
    \caption{Darstellung des Dispersionverlauf einer SPP-Welle und eines periodischen Goldgitters auf einer Probenoberfläche.}
    \label{fig:disp_und_gitter}
\end{figure}
\FloatBarrier

In Abbildung~\ref{fig:disp} ist die Dispersionsrelation für eine SPP-Welle und ein freies Photon zu sehen. 
Da es zwischen dem Verlauf von $k_\text{spp}$ und $k$ keinen Schnittpunkt gibt, kann
in diesem Fall keine Anregung eines SPPs erfolgen. 
Ein Schnittpunkt würde  einer energieerhaltenden Anregung entsprechen.
Der Prozess der Anregung eines SPPs ist auch umgekehrt möglich.

Befindet sich ein Goldgitter (vgl. Abbildung~\ref{fig:gitter}) 
mit Periodizität $a$ auf der Oberfläche so lässt sich 
ein SPP immer dann anregen wenn die Bedingung
\begin{equation}
    k_\text{SPP} = k \sin(\theta) \pm m \, G, \quad  m \in \mathbb{N}^+
\end{equation}
erfüllt ist.
Das SPP nimmt also nur einen Impuls in paralleler Richtung zur Oberfläche unter dem Winkel $\theta$ auf.
Es können beliebig viele verschiedene Ordnungen von
reziproken Gittervektoren $G = \sfrac{2 \, \pi}{a}$ hinzuaddiert werden.

\section{Große Zeemanaufspaltung}
Der Zeeman-Effekt beschreibt im allgemeinen das Phänomen der Aufspaltung von 
Energieniveaus beim anlegen eines äußeren Magnetfelds.
Das Bändermodell in einem Festkörper ist von dieser Aufspaltung ebenfalls betroffen.
Hinzu kommt das bei semimagnetischen Halbeitermaterialien eine sehr große Zeemanaufspaltung
entsteht.\cite{zeeman}

In Abbildung~\ref{fig:zeeman} ist die Zeemanaufspaltung schematisch dargestellt.
\begin{figure}
    \centering
    \includegraphics[scale=0.25]{./Plots/zeeman.png}
    \caption{Zeemanaufspaltung der unterschiedlichen Energiebänder bei äußerem Magnetfelds.\cite{felix}}
    \label{fig:zeeman}
\end{figure}
\FloatBarrier

Wird ein äußeres Magnetfelds angelegt, spaltet sich das Leichtloch~(engl. light hole, lh ) 
und das Schwerloch~(engl. heavy hole, hh) in jeweils 2 neue Niveaus auf.
Die Aufspaltung des Leichtlochs ist größer als die des Schwerlochs.
Ebenfalls entstehen 2 neue Niveaus des Leitungsbandes.
Die erlaubten optischen Übergänge und die dabei entstehenden unterschiedlichen 
Polarisationen des Lichts (beim Übergang) sind durch die Übergangsregeln festgelegt.
Bei den beiden blauen Übergängen in Abbildung~\ref{fig:zeeman} handelt
es sich um eine linkszirkulare Polarisation, also $\sigma^-$.
Bei den roten um eine rechtszirkulare, also  $\sigma^+$. 
Ein $\sigma$-Übergang bedeutet das ein $\Delta m$ der magnetischen Quantenzahl von 
$\pm 1$ angenommen wird.
Wird ein negatives Magnetfelds (linke Seite in Abbildung~\ref{fig:zeeman}) betrachtet,
dann ändern sich die Polarisationen von links- nach rechtszirkular und umgekehrt.
So ist beispielsweise, für negatives Magnetfeld, der ein Übergang von $S_\text{e,x} = \sfrac{-1}{2}$
nach $J_\text{h,x} = \sfrac{1}{2}$ linkszirkular Polarisiert .

\section{Magnetisch kontrollierte direktionale Lichtemission}
Die magnetisch kontrollierte direktionale Lichtemission
beschreibt wie mit Hilfe eines angelegten äußeren Magnetfelds, die Lichtemission
einer Halbleiterprobe gesteuert/geführt werden kann.
Das Prinzip ist, dass die Umpolung des Magnetfelds eine Änderung des SPP-Impulses zurfolge hat 
und dadurch die Lichtemission in eine andere Raumrichtung verläuft.
\begin{figure}
    \centering
    \includegraphics[scale=0.75]{./Plots/probe_komplett.pdf}
    \caption{Im Experiment verwendeter Probenkörper (vgl. Abbildung~\ref{fig:probe}).
    In den verschiedenen Schichten des Hybridhalbleiters entstehen Exzitonen und SPPs 
    welche miteinander koppeln. Am Goldgitter der Oberfläche wird das PL (roter Pfeil)
    emittiert.    
    Das angelegte Magnetfeld ist homogen und verläuft entlang der x-Achse der Probe.
    Die Richtung des Magnetfelds ist umkehrbar.}
    \label{fig:komplett}
\end{figure}
\FloatBarrier

Die in Abbildung~\ref{fig:komplett} zu sehende Halbeiterprobe befindet sich 
in einem homogenen Magnetfeld entlang der x-Achse.
Mit Hilfe eines Lasers werden im Quantentopf der Probe Exzitonen erzeugt.
Exzitonen beschreiben Quasiteilchen welche von einem Elektron-Loch-Paar gebildet werden.
Elektron-Loch-Paare entstehen z.B. wenn ein Laser 
Elektronen aus dem Valenzband in das Leitungsband hebt.\cite{jens}

Die entstehenden Exzitonen besitzen in der y-z-Ebene (vgl. Abbildung~\ref{fig:komplett}) eine 
rechtsseitige zirkulare Polarisation.
\footnote{Ob die zirkulare Polarisation recht- oder linksseitig 
ist kann im Detail nicht so einfach gesagt werden. 
Es geht darum, dass sich bei der Umpolung des Magnetfelds die Drehrichtung der Polarisation ändert.
Ebenso ist die Aussage, dass Exzitonen zirkular Polarisiert sind eher ungenau, denn die Polarisation
ist eine Eigenschaft des Lichtes bei der Rekombination zwischen Elektron und Loch.}
Diese Exzitonen können im folgenden, an die in der Deckschicht sich bildenden 
Plasmonen (SSP) mit der gleichen Polarisation, koppeln.
Bei anliegendem konstanten Magnetfeld entsteht eine Temperatur abhängige Vorzugspolarisation unter 
den entstanden Exzitonen.
%\footnote{Dieser Effekt ist Temperaturabhängig und wird später beschrieben.}
Sodass beispielsweise Anteilig mehr rechtszirkulare Exzitonen existieren
als linkszirkulare. 
Diese Differenz pflanzt sich in die Entstehung der Plasmonen (SPPs) fort.
Das sich auf der Oberfläche der Probe befindende Goldgitter, sorgt wie in Abschnitt~\ref{sec:spps}
beschrieben dafür, dass die SPPs in Photonen emittieren können und die Probe als PL verlassen.
Die gebildete Vorzugspolarisation sorgt bei diesem Prozess für eine ebenfalls bevorzugte Emittierung von Photonen
in eine Richtung. 
Dieser Effekt wird als transversale magnetische Führung der Lichtemission 
(engl. transverse magnetic routing of light emission, TMRLE) bezeichnet.
Wird die Temperatur der Probe um wenige Kelvin gesteigert, ist eine Abschwächung des Effekt zu erkennen,
denn die Erhöhung der Temperatur sorgt für eine Erhöhung der Anregungswahrscheinlichkeit für Exzitonen
mit der jeweils andersseitigen Polarisation. Formel?