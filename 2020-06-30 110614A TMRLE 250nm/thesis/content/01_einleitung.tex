\chapter{Einleitung}
Das Ziel dieser Arbeit ist die Untersuchung der Temperaturabhängigkeit
der transversalen magnetischen Führung von Lichtemission (engl.
transverse magnetic routing of light emission, TMRLE), anhand einer Halbleiter-Hybridstruktur.

TMRLE bedeutet, dass mit Hilfe eines angelegten Magnetfelds eine aus dem 
Probenkörper kommende Lichtemission (Photolumineszenzlicht), in
eine bestimmte Richtung geführt werden kann.

Physikalisch wird der Effekt hervorgerufen indem auf die hier verwendete 
Halbleiterprobe, welche sich in einem Magnetfeld befindet, ein Laserstrahl geschossen wird.
Der Laserstrahl erzeugt in dem Quantentopf der Probe Exzitonen, welche sich in der x-y-Ebene der
Probe befinden. Durch das Magnetfeld werden die Exzitonen teilweise in eine andere Ebene gekippt und
können infolgedessen an Plasmonen koppeln.
Plasmonen beschreiben kollektive Oszillationen des Elektronengases in einen Festkörper. 
Das sich auf der Probenoberfläche befindende Goldgitter, ist dafür verantwortlich 
das die Plasmonen in ein Photon emittieren können.
Das Magnetfeld ermöglicht die Emissionsrichtung der Photonen 
zu führen und es lässt sich die Größe der
Direktionaliät $C$ definieren, die diesen Prozess quantifiziert.

Wird die Probe extern geheizt, verkleinert sich die Direktionaliät $C$ infolgedessen. 
Diese Verringerung der Direktionaliät $C$, aufgrund der Erhöhung der Temperatur, wird in dieser Arbeit genauer untersucht.