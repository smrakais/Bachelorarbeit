\chapter{Einleitung}
Magnetooptische Effekte beschreiben, das Zusammenspiel zwischen Licht und Materie in magnetischen Feldern. 
Der erste bekannte Effekt und der Nachweis, dass eine gegenseitige Wechselwirkung überhaupt möglich ist, 
wurde im Jahr 1845 durch den englischen Experimentalphysiker Michael Faraday endeckt.~\cite{michi}
Erst einige Jahre später erfolgte dann die Beschreibung von Licht als elektromagnetische Welle.
Heutzutage gibt es mehrere magnetooptische Effekte, einer davon ist zum Beispiel der
transversale magnetooptische Kerr-Effekt (TMOKE). 
Bei diesem wird mit Hilfe eines Magnetfelds die Intensität von reflektiertem Licht verändert.
Der Effekt ist sehr klein und daher auch schwierig anzuwenden und zu messen. 
Er lässt sich aber um mehrere Größenordnungen verstärken, wenn man plasmonische Strukturen betrachtet.~\cite{kerr,kerr2} 

Der in dieser Arbeit untersuchte magnetooptische Effekt ist bekannt als transversale magnetische Führung von Lichtemission (engl.
transverse magnetic routing of light emission, TMRLE).
Wie 2018 gezeigt worden ist lässt sich auch die Intensität von emittiertem 
statt reflektiertem Licht mit plasmonischer Hilfe in der Geometrie des TMOKE räumlich steuern.~\cite{allg_paper}

Experimentell wird in dieser Arbeit die TMRLE hervorgerufen, indem auf eine plasmonische
Halbleiter-Hybridstruktur, welche sich in einem Magnetfeld befindet, ein Laserstrahl geschossen wird.
Der Laserstrahl erzeugt in dem Quantentopf der Probe Exzitonen, welche sich in der x-y-Ebene der
Probe befinden. 
Durch das Magnetfeld werden die Exzitonen teilweise in eine andere Ebene gekippt und
können infolgedessen an Oberflächenplasmonen koppeln.
Diese Oberflächenplasmonen beschreiben kollektive Oszillationen des Elektronengases in einem Festkörper. 
Ein sich auf der Probenoberfläche befindendes Goldgitter ist dafür verantwortlich, 
dass die Plasmonen als ein Photon emittieren können.
Die Umpolung des Magnetfelds ermöglicht die Emissionsrichtung der Photonen 
zu kontrollieren.

Diese Form der magnetisch kontrollierten direktionalen Lichtemission ist auch für Anwendungen interessant, 
weil durch das Anpassen des 
plasmonischen Gitters und der Probeneigenschaften die Charakteristik der Emission individuell angepasst werden kann, 
also zum Beispiel der Winkel des maximalen Effekts und die Wellenlänge. 
Um zukünftige, einfachere Anwendungen denkbar zu machen, 
muss der Effekt jedoch auch bei höheren Temperaturen als bei der von flüssigem Helium funktionieren.

Aus diesem Grund ist das Ziel dieser Arbeit die Untersuchung der Temperaturabhängigkeit
der TMRLE anhand einer Halbleiter-Hybridstruktur.